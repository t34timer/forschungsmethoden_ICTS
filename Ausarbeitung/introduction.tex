\section{Introduction}
\label{sec:introduction}
%Einleitung
%Warum ist das Thema wichtig?
%Wo wirds benötigt oder eingesetzt?
%Welche Fragen sollen beantwortet werden?

For software development projects it is essential to use proper software tools to manage historical information of a project \cite{fischer2003populating}. The use of version control systems like SVN, Git, Mercurial, etc. provide features to track changes on the developed software. Changes on software are made to upgrade, fix or replace parts of the software to improve the overall system \cite{janak2009issue}. However, changes do not always fulfill the intended goal of improving the system or get deprecated after some time. Therefore, it is important to collect information about all changes committed in a repository. The information that can be extracted of version control systems are e.g. change rates, error proneness or starvation of the software project \cite{fischer2003populating}. This can help to understand the evolution of the project to learn from past development mistakes and improve the development in the future.

The software changes in a project are based on specific issues like error reports, feature requests or requirement changes. Those issues can be tracked by bug repositories or issue tracking systems \cite{fischer2003populating}. Especially in the past years, there is an increasing attention of reporting bugs in issue tracking systems \cite{ahmedpredicting}. This is a positive evolution, as developers get active feedback of users and testers, which can result in a faster elimination of problems. 
Every issue in such a management tool contains specific meta information about the problem that has to be solved. I.e. for example a description of the problem, the time to solve the issue and an impact analysis, meaning a possible solution to fix the problem \cite{fischer2003populating}. The state of the issue can be updated by the according actors that resolve the issue, which enables stakeholders to track the progress. Information like the issue title, description, impact analysis is reported with free text or natural language \cite{ahmedpredicting}. Hence, there is a large number of possibilities to process the information and classify issues based on previous analyzed meta information. Known data classification techniques are e.g. Naïve Bayes, Support Vector Machine (SVM), Decision Trees, K nearest classifiers and Neural Networks \cite{ahmedpredicting}\cite{lancaster2003indexing}. I.e. a new issue can be assigned to a specific category upon creation. This can help the project manager to assign the issues to the appropriate developer. Furthermore, the developers can identify relevant parts of the code for their specific tasks \cite{ying2004predicting}.

As the version control repositories and the issue tracking systems hold significant information about the project development, the combination of both systems can help to better understand the evolution of the software. E.g. when a developer commits multiple changes to the software repository it can be hard to retrace the intention of those changes after some time. Even if all commits have a short description explaining the changes it might not be clear which issue is related to software changes. Applying natural language analyzation methods on commits and issues can help to classify and connect either, without the need to specify an issue number in every commit message.

In this paper we want to present a method to classify information from issue tracking systems. We developed a program that applies the method on sample data of a software project. The sample data was extracted from a SVN repository and a Jira issue tracking system. All information of the sample data was anonymized to protect the data privacy of the software project contributors.