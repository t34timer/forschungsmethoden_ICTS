\section{Methods}
\label{sec:methods}

In order to find a relation between the commit messages and the issues, we need to start by reading the input data and tokenizing each commit message and the title and description of each issue into tokens. In order to properly analyze natural languages, we need to lemmatize the tokens. Lemmatization makes it possible to detect words that have the same meaning but appear differently in the text, either because of its tense (like ''walk'' and ''walked'') or because of the nature of that word (like ''good'' and ''better''). Lemmatization normalizes every word so that we have one token for every word with the same meaning. After the lemmatization step, we need to remove stop words so that only the interesting tokens are left. The resulting tokens will then be placed in a table where the frequency of each token is stated for each commit message and issue. Using this frequency table of tokens, we can compare the tokens of commit messages with the tokens of issues and find out the similarities based on the common tokens.


We also have access to the date, theme and the author for both commit messages and issues. This kind of information can be useful when matching issues with commit messages. For example, if a commit is done after an issue was submitted, it is less likely that the cause of this issue is related to that commit. Similarly the theme and the author of commits and issues can help us classify the input and provide better results. Grouping the commits and issues by their themes can increase the accuracy of matching and when the common words are not enough to determine a match, the system can provide a set of commits with a similar theme to the issue, instead of a single matching commit. This might make it easier to find the commit manually, since the user has to only search a portion of the commits.
